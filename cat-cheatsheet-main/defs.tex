\usepackage{amsmath}
\usepackage{amsthm}
\usepackage{amssymb}
\usepackage{stmaryrd}
\usepackage{mathrsfs}
\usepackage{mathtools}
\usepackage{tikz-cd}
\usepackage{quiver}

\newcommand{\cat}[1]{\mathscr{#1}}
\newcommand{\ob}[1]{\mathrm{ob}(\cat{#1})}
\DeclareMathOperator{\op}{op}
\newcommand{\map}[1]{\xrightarrow[]{#1}}
\newcommand{\maps}[2]{\begin{tikzcd}[ampersand replacement=\&,cramped, sep=scriptsize]
	{} \& {}
	\arrow["#1", shift left, from=1-1, to=1-2]
	\arrow["#2"', shift right, from=1-1, to=1-2]
\end{tikzcd}}

\newcommand{\isofuncs}[2]{\begin{tikzcd}[ampersand replacement=\&,cramped, sep=scriptsize]
	{} \& {}
	\arrow["#1", shift left, from=1-1, to=1-2]
	\arrow["#2", shift left, from=1-2, to=1-1]
\end{tikzcd}}

\newcommand{\ladj}[2]{\begin{tikzcd}[ampersand replacement=\&,cramped, sep=scriptsize]
	{}  \& {}
	\arrow[""{name=0, anchor=center, inner sep=0}, "#1", shift left=2, from=1-1, to=1-2]
	\arrow[""{name=1, anchor=center, inner sep=0}, "#2", shift left=2, from=1-2, to=1-1]
	\arrow["\dashv"{anchor=center, rotate=-90}, draw=none, from=0, to=1]
\end{tikzcd}}

\DeclareMathOperator{\Set}{\mathbf{Set}}
\DeclareMathOperator{\Cat}{\mathbf{Cat}}
\DeclareMathOperator{\CAT}{\mathbf{CAT}}
\newcommand{\adj}[0]{\dashv}
\newcommand{\comma}[0]{\downarrow}
\DeclareMathOperator{\Hom}{Hom}
\DeclareMathOperator{\iso}{\xrightarrow[]{\sim}}
\newcommand{\scat}[1]{\mathbf{#1}}
\DeclareMathOperator{\Cone}{Cone}
\DeclareMathOperator{\limcat}{\lim_{\leftarrow \scat{I}}}